\documentclass{report}\usepackage[]{graphicx}\usepackage[]{color}
%% maxwidth is the original width if it is less than linewidth
%% otherwise use linewidth (to make sure the graphics do not exceed the margin)
\makeatletter
\def\maxwidth{ %
  \ifdim\Gin@nat@width>\linewidth
    \linewidth
  \else
    \Gin@nat@width
  \fi
}
\makeatother

\definecolor{fgcolor}{rgb}{0.345, 0.345, 0.345}
\newcommand{\hlnum}[1]{\textcolor[rgb]{0.686,0.059,0.569}{#1}}%
\newcommand{\hlstr}[1]{\textcolor[rgb]{0.192,0.494,0.8}{#1}}%
\newcommand{\hlcom}[1]{\textcolor[rgb]{0.678,0.584,0.686}{\textit{#1}}}%
\newcommand{\hlopt}[1]{\textcolor[rgb]{0,0,0}{#1}}%
\newcommand{\hlstd}[1]{\textcolor[rgb]{0.345,0.345,0.345}{#1}}%
\newcommand{\hlkwa}[1]{\textcolor[rgb]{0.161,0.373,0.58}{\textbf{#1}}}%
\newcommand{\hlkwb}[1]{\textcolor[rgb]{0.69,0.353,0.396}{#1}}%
\newcommand{\hlkwc}[1]{\textcolor[rgb]{0.333,0.667,0.333}{#1}}%
\newcommand{\hlkwd}[1]{\textcolor[rgb]{0.737,0.353,0.396}{\textbf{#1}}}%

\usepackage{framed}
\makeatletter
\newenvironment{kframe}{%
 \def\at@end@of@kframe{}%
 \ifinner\ifhmode%
  \def\at@end@of@kframe{\end{minipage}}%
  \begin{minipage}{\columnwidth}%
 \fi\fi%
 \def\FrameCommand##1{\hskip\@totalleftmargin \hskip-\fboxsep
 \colorbox{shadecolor}{##1}\hskip-\fboxsep
     % There is no \\@totalrightmargin, so:
     \hskip-\linewidth \hskip-\@totalleftmargin \hskip\columnwidth}%
 \MakeFramed {\advance\hsize-\width
   \@totalleftmargin\z@ \linewidth\hsize
   \@setminipage}}%
 {\par\unskip\endMakeFramed%
 \at@end@of@kframe}
\makeatother

\definecolor{shadecolor}{rgb}{.97, .97, .97}
\definecolor{messagecolor}{rgb}{0, 0, 0}
\definecolor{warningcolor}{rgb}{1, 0, 1}
\definecolor{errorcolor}{rgb}{1, 0, 0}
\newenvironment{knitrout}{}{} % an empty environment to be redefined in TeX

\usepackage{alltt}

\usepackage{natbib}
\usepackage{hyperref}

\makeindex
\IfFileExists{upquote.sty}{\usepackage{upquote}}{}

\begin{document}

\author{Ross T. Harrison}
\title{Lab Journal}
\date{Spring 2014}
\maketitle

\begin{abstract}
This is a journal of notes and experiments conducted as part of the requirements for ENGLISH 4/898 - SEC 002 \emph{Characterization in Literature: a Macroanalysis}
\end{abstract}

\section{Structural Analysis on Social Networks Constructed In Literature Texts(Park)}
\date{January 18, 2014}

\subsection{Goals}
Use techniques applied to economics, sociology, and other disciplines on literature. The purpose is to measure connectivity between characters.

Previous work has investigated social structure in novels as a static whole (Elson). Their eventual goal is to investigate the change in social structure over time. The goal of this paper is to demonstrate the effectiveness of their co-referencing technique

\subsection{Technique}
Their technique is similar to previous work. Using the textual distance between mentions of a character to calculate the weight of their connection. A key difference with their algorithm is they use distance in statements instead of words. That is if two characters are in the same sentence then their distance is 0. If in adjacent sentences their distance is 1.

The weight is then calculated from the distance using a power function. This is nice because it has a maximum value (1) and rapidly (depending on variable) approaches 0, but never reaches. The relationships then have a maximum weight, but are never "meaningless". For computational purposes however, the group decided to enforce an influence region restriction, or how many statements ahead will the algorithm look. They used 10 as their influence region.

\subsection{Conclusions}
Using their work as a guide could be useful, but the paper is too poorly written, ill defined, and data delivered in to low a resolution, to investigate their exact results.

However, their idea of investigating changes in networks over time is interesting. However, if I were to pursue it, I would like to create a "real time" viewer that would allow the user to scrub the piece, and see the relationships change fluidly.

\section{Character Social Networks}
\date{January 26, 2014}

A reocurring idea is that character archetype can at least in part be determined by relations with other characters. This problem can be broken into several subproblems, first would be determining character existence, second the existance of a relationship between those characters, and thirdly the nature of that relationship. 

The first subproblem is the most quantifiable. Since for many novels a complete, or semi-complete list of characters is available. Success metrics are therefore readily available. The second and third seem more difficult. Defining a correct answer for character relationship existence, much less quality, is much more difficult, if not intractable. 

Relationship existance at least seems to have a possible solution, comparing with human created character networks. This has the obvious drawback of being extremely labor intensive in constructing a test set. Therefore, unless such a corpus is readily available, a prudent course of action might be to investigate network creation in a more qualitative manner.

\section{Character Network Construction}: (January 26, 2014)

Character networks can be constructed using dialogue (Elson, Dames McKewon)
Or by simple coocurence in text (Park, or LOTR project)

Like Park et-al, I'm interested in investigating social changes over time. However, while they intend to divide the text into chapeters, I believe that a more interesting view would be to construct a social graph for any given point in the text. 

My idea is create a web interface that allows the user to upload text and a character list. Then produces a network similar to the one seen on the LOTR project. However, this graph will allow the user to adjust constants.

The user will also be able to "scrub throught the text" viewing the changes in real time. Using statements as the atomic units, the connections made would be further weighted by distance from current position using some kernel. A variety of options could be available:
* A power function
* a normal function

The idea then is to see in "real time" the changes in importance and strength of character relationships.

Exposing the constants and filters to the end users

\section{Elson}
\date{January 27, 2014}

This article was referenced by park. It is looking at two claims made
about 19th century brittish novels:

\begin{enumerate}
\item
  there is an inverse correlation between coversation and \# of chars
\item
  There are more interactions occuring in rural, rather than urban settings
  
\end{enumerate}

They intend ot construct networks based on dialogue. First finding
quoted speech and attributing to a set of characters. Edge weights are
determined by the ammount of conversation going on between the
characters.

60 novels, 31 authors

They claim that this study contradicts widely held notions

\subsection{Related Works}

Burrows 2004, word usage patterns. Mostellar and Wallace. ``outing''
authors Lee, 2007. Validating lineage of ancient texts

Semantic analysis has been more rare. Exceptions: Chambers and Jurafsky
2008

Moretti (2005) graphically mapped out text according to: * geography, *
social connection * etc

\subsection{Hypotheses}

Statistical methods are essetial for testing validity of core theories.

Bakhtin 1981, 84. Different spatial settings have different
potentialities and govern social interactions in a way that should turn
up in analysis. -\textgreater{ Rural communities are smaller tighter
nit groups than urban. Moretti (1999) argues, horizontal connection are
more important in city than transgeneration. As the number of characters
increases the social connection become more complex and the whole system
becomes unstable, and blur lines between character roles.

Novels are divided into urban, rural, and mixed groups. Presumably
manually. They were selected as representative based on cannonicle
authers, decade, setting, and sub genre.

\begin{itemize}
\item
  Urban - set in metro, multiple labor forms.
\item
  Rural - set in village, agriculture is primary, landowning gentry are prominent.
\end{itemize}

Pulled from Guttenburg

\subsubsection{Hypotheses stated}

\begin{enumerate}
\item
  Inverse Correlation Between \# of characters and the ammount of dialogue
\item
  19th cent brit lit. depicts urban groups and large and loosely related. rural as tightly bound smaller.
\end{enumerate}

These two hypotheses are potentially related, as there are generally
more people in cities.

\subsection{Extracting Conversation Networks}

Conversation: 
\begin{enumerate}
\item
  continuous span of narrative
\item
  characters in same place at same time
\item
  characters take turns speaking
\item
  characters are mutually
\item
  aware of each other's speech, and are intended to hear.
\end{enumerate}

\subsubsection{Character Identification}

First each novel was processed with Stanfor Named Entity Recognition
tagger (Finkle 2005). Noun phrases extracted that were categorized as
person or oganzation. Nouns clustered into entities.

Clustering: 1. For each Named entity. 1. Generate variants *
Mr.~Sherlock Holmes, Mr.~Holmes, Sherlock Holmes, etc 2. Compile list of
coreferent names, variations etc 2. Aggregate mentions

Text was also preprocess to nomalize formatting and detect sections,
especially looking for quated speech.

\subsubsection{Quoted Speech Attribution}

Approach Described in previous workd (Elson \& McKeown, 2010). Training
set of Brit, American and Russian Texts. Amazon Mechanical Turk provided
human text cases. Accuracy was like 83\%, but because conversation is
what is desired, not quotes, success is likely higher. Precision
emphasized over recall. That is the emphasis is on the percentage of
valid answers not on retrieving all valid answers.

\subsection{Evaluation}

Four novels were held out of the training set. And only used to
evaluate. For each novel random chapters were selected, mannually
processed. Multiple conversations were counted as individual components.

Speech Adjacency has high precision (.95) recall 0.51 (almost half the
results were missing). correlation and spoken mention methods
(alternative edge creation methods) had much lower precisions (0.21,
0.45), but simimilar recall (0.21, 0.45).

\subsection{Data Analysis}

\subsubsection{Feature Extraction}

\begin{enumerate}
\item
  Number of characters and Number of speaking characters (seperate features?)
\item
  The variance of the distribution of quoted text among n most frequent features for n between 1 and 5
\item
  Number of quotes and percentage of novel as quoted speech
\item
  Number of 3-cliques and 4-cliques in social network
\item
  Average Degree of the graph, that is the average number of edges per
  node
\item
  Graph density, normalizing the average degree by number of characters: 2.
  The idea is to capture what percent of the entire network does each character average.
\end{enumerate}

Weight of edge over 0 doesn't affect feature 5 or 5.

\subsubsection{Results}

Pearson's product-moment correlation.

Hypothesis 1 - They found a weak positive correlation. Stronger
correlation between number of unique speakers and number of quotes.
Also, the connectedness (average degree) of the graph had a positive
correlation with \# of chars. Analysis suggests the opposite is true,
small communities tend to be more disconnected.

Hypothesis \#2 - No significant difference in the size of the graphs.
Simply ``not confirmed''

However, third person narrative, does have more frequent connectivity.
This makes sense since monologue isn't really a huge deal.

Aside, it is interesting to think about ASOIF in this case, since it seems to mix first and third person. The character perspectives shift, and language is third person, the reader is bound by the characters perspective and thoughts.


\subsection{Literary Conclusion}

Narrative Voice trumps setting.

\subsection{conclusion}

High precision for detecting face to face communication between two
named characters. Narrative perspective is a much stronger predictor of
graph characteristics
\section{Stanford Papers (pulled from their NLP site)}

\subsection{Baselines and Bigrams: Simple, Good Sentiment and Topic Classification}
\date{January 30, 2014}

Investigating performance and efficiency of Naive Bayes (NB), SVM, and Hybrid solutions on Sentiment tasks. NB actually does better. Hybrid can provide new state of the art performance level.

Data set includes customer reviews, movie reviews, IMDB, newsgroups.

Tokenization was used when available. If not unigrams were created. Non Alphabetical were filtered out. Cross validations splites were observed in several cases. Delta values for p < 0.5 were displayed. 

MNB better at snippits. Statistical methods miss some cases in smaller sets. ``Not An Inhumane Monster'', ``Killing Cancer''.  

SVM is better at full-length reviews. When the target of sentiment in a larger body of text then, SVM performs better.

NVSVM espcicially bigram features performed well across the board, large and short sections. ``Often gives better results than previously published'' (trigrams hurts slightly). 

Utility? See what features are being used for sentiment analysis when it comes to it. Need to learn more about what features are in the SVMs in general.

\
\section{Character Profile: Bobby Newmark \textit{Count Zero}}
\date{February 2, 2014}

Bobby Newmark is an aspiring cyberspace cowboy. From the very beginning, Bobby is helpless. He is introduced near the moment of his death and is only saved through the intercession of an artificial intelligence. It is then revealed that the cyberspace deck (computer) he used to do is run (hack job) is borrowed. He has a handmedown a holographic pornography projector that he uses to give his room "a sense of space". Even his memories aren't his. He has flashbacks related to a show that his mother watched while he was in Utero. The nature of the delivery has left him with ghost memories. He lives in a slum called Barrytown where nobody has much. His outfit is a simple black tshirt with black jeans, and sandals.

Bobby Newmark is the personification of dehumanization that is characteristic of Cyberpunk, and Gibson. His things, look, and identity are of no concequence, nor are they his own.

Pulling this out of the structure could be difficult if Bobby wasn't the only character in the first couple of scenes. Lack of resources, and agency could be tough to put together. However modeling them as X does something to Y, could be effective. Since things are always happening to Bobby. At least in the begining. 

\date{February 5, 2014}
As the novel shifts into its second phase, bobby becomes ware of his inferiority.

paraphrases
\begin{enumerate}
\item
  ``Here I am with a serious operator (crime muscle) and I pull a total Wilson (mistake)''
\item
  ``'Was he stupid?' Bobby immediately regretted the question.''
\item
  ``Lucas stood with Bobby directly in front of him, like a small child.''
\end{enumerate}

Obviously the exact meaning of these phrases, and analyzing them in context would be a monumental undertaking. However, I think that because language like that appears near Bobby frequently could show up in a topic model. 

Because this novel seperates naratives it could have similar qualities as A Song if Ice and Fire. However, I haven't reached the point where the characters are in the same room. So I don't know if the strict narrative limits will be followed as in ASOIF, or not.

\section{Latent Diriclet Allocation}
\date{February 5, 2014}

\textit{reading wikipedia article.}

This came up when reading about topic modeling, and Dr. Jockers had mentioned previously.

Oh hey cool, Andrew Ng is involved.

It seems to me that running this kind of analysis would require a large set of topic related words before analysis can begin. This would be fine for a small corpus, but the analysis of a diverse one presents a problem of where these topics are retrieved. Is there some preexisting database?

Unsure how it works. Will have to run or see example to understand.

``The Bayesian formulation tends to perform better on small datasets because Bayesian methods can avoid overfitting the data.'' Seen previously in the standfor paper on baselines and bigrams.

Has been applied to images. 
\ur{"http://machinelearning.wustl.edu/mlpapers/paper_files/NIPS2007_102.pdf"}

\bibliographystyle{apa}
\bibliography{journal}

\end{document}
